\chapter{Etyczne i prawne aspekty web-scrapingu}

\section{Wprowadzenie}

	Web scraping ze względu na swoje działanie jak i późniejsze wykorzystywanie pozyskanych informacji traktowany jest zwykle jako procedura wątpliwa nie tylko moralnie, ale często również jako działanie nie do końca zgodne z prawem. Mamy tu przecież do czynienia z pozyskiwaniem, zwykle bez wiedzy właściciela strony, portalu czy serwisu, informacji, które mogą być objęte prawami autorskimi, naruszającymi czyjąś prywatność lub wręcz tajnymi (ale nie dostatecznie zabezpieczonymi).

\section{Scraping w życiu codziennym, regulacje prawne}

Podstawy prawne, które mogą regulować zasady działania web-scrapingu różnią się w zależności od naszego położenia na kuli ziemskiej. Ogólnie rzecz biorąc, takie wytyczne mogą być ujęte w zasadach korzystania z danej strony internetowej. Przestrzeganie tychże zasad i dochodzenie swoich praw ze strony „poszkodowanego” portalu nie zawsze jest jasne i łatwe do przeprowadzenia. Biorąc pod uwagę powyższe, niektóre kraje zdecydowały się wprowadzić regulacje, które wprost decydowałyby o tym, co jest, a co nie jest legalne. Jednakże, ze względu na różnorodność jaka charakteryzuje działania związane ze scrapingiem ostateczna decyzja należy zwykle do sądu. Mowa tu oczywiście o działaniach, których nie można wprost określić jako nielegalnych np. kradzieży danych i wykorzystywania ich w negatywnych celach. Spory kończące się na ławach sądowych dotyczą zwykle wielkich korporacji działających przede wszystkim w sieci takich jak Facebook, Google, LinkedIn, ale również firm działających w różnych gałęziach gospodarki, ale w obecnych czasach nierozerwalnie związanych już z Internetem. 
	
Ponieważ spraping wykorzystywany jest przez strony dzięki którym można na przykład porównywać ceny biletów lotniczych, sprawdzać dostępność hoteli, ceny pokoi czy chociażby porównywać ceny dostępnych produktów w bezmiarze sklepów internetowych istniejących obecnie. Z punktu widzenia konsumenta działania takie są jak najbardziej pozytywne, jednak z punktu widzenia sprzedawcy już nie tak bardzo. Te same dane, zamiast służyć dobru konsumenta, wykorzystane mogą być również przez inne firmy do nie do końca czystych zagrań, jak np. celowe obniżanie cen. 

Inny negatywny aspekt zauważyć można chociażby w przypadku odpłatnych imprez masowych. Dotyczy to zjawiska „automatycznego podkupowania” biletów przez inne podmioty oczywiście w celach uzyskania dodatkowych korzyści z późniejszej ich sprzedaży po swoich cenach. Aby temu przeciwdziałać wprowadzono ograniczenia w ilości kupowanych biletów przez pojedynczego użytkownika. Nie jest to rozwiązanie idealne, ale też nie uderza zbytnio w konsumenta zainteresowanego kupieniem biletu w najlepszej możliwej cenie. 

\section{Podsumowanie}

Scraping w różnych formach istniał praktycznie od momentu rozpowszechnienia się internetu i zapewne istniał będzie aż po jego kres (albo i dłużej). Biorąc pod uwagę zalety, które przynoszą takie działania ciężko stwierdzić czy zaistnieje kiedykolwiek wykładnia, która raz na zawsze zabroni jednym podmiotom korzystać z danych zamieszczanych w sieci przez innych. 

Jak pokazuje historia, każdy przypadek wchodzenia na drogę sądową przez firmy „dotknięte” działaniem scrapingu jest inny i bez ścisłych uregulowań prawnych nie do przewidzenia jest decyzja jaka zostanie podjęta przez sąd rozpatrujący daną sprawę. 