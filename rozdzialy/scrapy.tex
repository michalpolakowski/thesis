\chapter {Scraper} 

W niniejszym rozdziale zajmę się omówieniem użytych w projekcie zagadnień dotyczących webscrapingu.

Scrapy jest jedną z najpopularniejszych bibliotek pythonowych, które mają za zadanie zbierać informacje ze stron internetowych.
Webscraping oparty jest na pyhtonowym frameworku \emph{Scrapy}, wspieranym przez biblioteki \emph{Splash} oraz \emph{XPath}.
Uzyskiwałem wiedzę posługując się dokumentacjami bibliotek \emph{Scrapy}\footnote{\url{https://doc.scrapy.org}}, \emph{Splash}\footnote{\url{https://splash.readthedocs.io/en/stable/api.html}} oraz \emph{XPath}\footnote{\url{https://doc.scrapy.org/en/xpath-tutorial/topics/xpath-tutorial.html}}.

Skrypt uruchamiany jest na trzech stronach sklepów internetowych: \emph{https://reserved.com, https://zalando.pl, https://domodi.pl}. 
Wykonywanie programu jest podzielone na trzy etapy:
\begin{enumerate}
	\item Po uzyskaniu słów kluczowych program wprowadza w wyszukiwarce strony daną frazę wpisaną przez użytkownika.
	\item Przekierowuje do wszystkich wyników wyszukiwania.
	\item Zbiera oraz zwraca informację z rubryki wygenerowanej dynamicznie przez stronę, która	ma na celu poinformowanie kupującego o rzeczach zakupionych przez innych użytkowników wraz z wyszukiwanym ubraniem.
\end{enumerate}
\section{Opis etapów}

\subsection{Etap 1}
	Aplikacja włącza program scrapingowy za pomocą następującej komendy: \emph{scrapy crawl clothing -a tag=”fraza” -o results.json}.
Oznacza to uruchomienie programu o nazwie \emph{clothing} z parametrem do wyszukania (tag), a rezultat będzie przechowywany do pliku \emph{results.json}.
Po uruchomieniu, do każdej strony internetowej inicjowana jest metoda \emph{search}, która za pomocą \emph{XPath} odszukuje formularz na stronie głównej sklepu, wpisuje interesującą użytkownika nazwę ubrania i wysyła request z żądaniem wyszukania.

\subsection{Etap 2}
	Strona sklepu internetowego zwraca wyniki wyszukiwania. Program zbiera wszystkie linki do wyszukanych ubrań za pomocą \emph{XPath}, który przeszukuje w kodzie HTML odnośniki mające w adresie przekierowującym rodzaj ubrania. Następnie przechodzi do tych stron odpalając kolejną metodę, która zależnie od sklepu szuka w kodzie HTML strony innych struktur.

\subsection{Etap 3}
	Będąc na tym etapie program jest już w dokładnie jednej ofercie zakupu ubrania. Zależnie od strony, skrypt zachowuje się w inny sposób:
\begin{itemize}
	 \item dla stron \emph{reserved} oraz \emph{domodi} od razu następuje przeszukiwanie kodu HTML w celu znalezienia proponowanych rzeczy poprzez znalezienie odpowiedniej struktury
		\begin{itemize} 
			\item dla sklepu Reserved każda oferta przeszukiwania jest umieszczona w znaczniku \textless div\textgreater, a znacznik ten zaczyna się od klasy, która ma nazwę \emph{portrait}. Dla Domodi jest to każdy 	znacznik \textless li\textgreater.
		\end{itemize}
	\item dla sklepu Zalando wyszukiwany jest najpierw tekst \emph{zobacz więcej}. Po znalezieniu tekstu aplikacja przechodzi na tą podstronę, żeby móc wylistować wszystkie ubrania proponowane przez sklep. W przypadku, gdy tekst \emph{zobacz więcej} nie zostanie znaleziony, aplikacja nie zwróci żadnych propozycji ze strony Zalando.\end{itemize}
Program pobiera następujące informacje z propozycji kupna:
\begin{itemize}
	\item adres URL
	\item zdjęcie ubrania
	\item cenę
	\item nazwę ubrania
\end{itemize}
Informacje zapisywane są tylko wtedy, jeśli jest ich cały komplet – powoduje to uniknięcie przedostawania się do wyników niepotrzebnych danych bądź innych przypadkowo pobranych rzeczy.
Zapisywanie odbywa się w formacie json o strukturze słownikowej:
\begin{itemize}
\item[] \emph{\{ClothesName: \{ "image": imageURL, "price": PRICE, "url": URL\}\}}
\item[] \emph{ClothesName} – przechowywany w formacie string; jako wartość zawiera słownik ze szczegółami
\item[] \emph{imageURL} – przechowywany w formacie string; zawiera adres do zdjęcia ubrania
\item[] \emph{PRICE} – przechowywany w formacie string; zależnie od strony może być z końcówką „zł” lub „PLN”
\item[] \emph{URL} – przechowywany w formacie string; zawiera adres do strony sklepu z daną rzeczą
\end{itemize}
Program na bieżąco przesyła taki słownik do pliku wynikowego, z którego dalsza część programu może na bieżąco czytać.


Program, żeby mógł działać na dynamicznych stronach używa biblioteki Splash.
Splash używa funkcji napisanej w języku Lua\footnote{\url{https://splash.readthedocs.io/en/stable/scripting-tutorial.html}}:
\begin{lstlisting}
function main(splash, args)
          assert(splash:go(args.url))
          assert(splash:wait(0.5))
          assert(splash:set_viewport_full())
          return {
            html = splash:html()}
            end
\end{lstlisting}

\begin{itemize}
\item[] Splash:go(args.url) przechodzi na stronę podaną w argumencie.
\item[] Splash:wait(0.5) – określa jak długo program czeka na załadowanie strony
\item[] Splash:set\_viewport\_full() sprawia, że wczytywana jest cała strona HTML

Wysyłanie requestów do stron również odbywa się za pomocą metody z biblioteki Splash -- \emph{SplashRequest}, która jako argumenty przyjmuje adres strony, metodę do której ma przejść w następnym kroku oraz argumenty (przekazanie całej funkcji z języka Lua jako parametr).
\end{itemize}
\section{Dlaczego Scrapy}

Scrapy jest stosunkowo dużym frameworkiem skupionym na webscrapingu. Nie jest trudno opanować podstawy -- framework ten jest bardzo dobrze udokumentowany oraz posiada dużo poradników dotyczących pierwszych kroków. Wszystkie te udogonienia pozwalają poznać podstawowe fundamenty Scrapy, które w dużej mierze powinny wystarczyć do większości zadań związanych z webscrapingiem. Posiada własne selektory, dzięki którym można swobodnie poruszać się po stukturze HTML. Jego największą zaletą jest to, że służy jako crawler -- wystarczy podać adres początkowy, a framework będzie w stanie przeszukiwać wgłąb stronę wedle chęci osoby piszącej program.

Przy wielkości opisywanego przez nas projektu oraz przy daleko idących planach rozbudowy aplikacji pisanie crawlera we frameworku w pełni skupionym na scrapingu jest niezbędne do uniknięcia dalszych problemów związanych z obsługą coraz bardziej skomplikowanych stron internetowych.
\subsection{Selektory}

Scrapy implementuje własne selektory pozwalające przeszukiwanie kodu HTML.
Można wydobywać części kodu za pomocą języka XPath oraz poprzez wybieranie kodu CSS, który stylizuje HTML.

W opisywanej aplikacji zdecydowałem się na posługiwanie się językiem XPath -- był dla mnie wygodniejszy i łatwiejszy w nauce.
Przykładowy selektor z napisanego przeze mnie scrapera, który jest używany w naszym programie:
\begin{lstlisting}
response.xpath("//a[./img[contains(@src,.)]]/@href").extract()
\end{lstlisting}
Powyższa linijka oznacza wybranie wszystkich znaczników \textless a\textgreater. Następnym krokiem jest znalezienie takich znaczników \textless img\textgreater, które posiadają w sobie odnośniki do innych stron. 
\emph{Extract()} powoduje, że wszystkie wyniki wyszukiwania pojedynczo będą formatowane jako typ string, który będzie można przekształcić wedle uznania.
Ważnym elementem selektorów jest możliwość wyszukania danej informacji znajdującej się wewnątrz wybranych znaczników.
Za przykład posłuży mi fragment kodu. Ma on za zadanie zebrać wszystkie niezbędne informacje na temat strony \emph{zalando.pl}:
\begin{lstlisting}
        for item in response.xpath("//div[./a[contains(@href,'zalando.pl')]]"):
            url = item.xpath('./a/@href').extract()
            itemImg = item.xpath('.//*/img[contains(@src,.)]/@src').extract()
            price = item.xpath(".//span[contains(text(),'zl')]/text()").extract_first()
            itemName = item.xpath('.//*/img[contains(@src,.)]/@alt').extract()
\end{lstlisting}

Kod ten napisałem sam, za pomocą dokumentacji Xpath oraz Scrapy. Xpath zaczynający się od kropki oznacza, że program zaczyna przeszukiwać od -- podanego w przykładzie wyżej, \textless a\textgreater. Wybierane są tylko hiperłącza posiadające odnośnik do strony zawierającej podstring "zalando.pl".

Finalnie, jeśli zdajemy sobie sprawę które rzeczy Scrapy powinien szukać oraz mamy otwartą stronę z dokumentacją\footnote{\url{https://doc.scrapy.org}}, to napisanie owego crawlera od podstaw, nie znając wcześniej składni, okazuje się stosunkowo proste i przyjemne.
