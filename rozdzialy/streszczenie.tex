
Nasza praca opisuje aplikację webową służącą jako pomoc w dopasowaniu za pomocą web-scrapingu elementu garderoby do najlepszych propozycji dostępnych w sklepach internetowych. 

Naszym celem było ukazanie w jaki sposób można zaimplementować web-scraping w aplikacji internetowej.

Web-scraping oparty jest na frameworku Scrapy, aplikacja zbudowana jest przy użyciu frameworka webowego Django, anonimizacja użytkownika i optymalizacja wydajnościowa jest możliwa dzięki serwerowi NGINX. Wszystkie zadania są obsługiwane przy użyciu kolejek dzięki Celery, a zbierane informacje są zapisywane w MongoDB oraz PostgreSQL. W pracy opisujemy działanie każdego z modułów, oraz przedstawiamy moralne aspekty web-scrapingu.