\documentclass[12pt,a4paper,twoside,openany,parskip=full,parindent=full]{book}
\usepackage[T1]{fontenc}
\usepackage[utf8]{inputenc}
\usepackage[polish]{babel}
\usepackage{graphicx}
\usepackage{times}
\usepackage{indentfirst}
\usepackage[left=3cm,right=2cm,top=2.5cm,bottom=2.5cm]{geometry}
\usepackage{natbib}
\usepackage{color}
\usepackage{tikz}
\usepackage{url}
\usepackage{dirtree}
\edef\restoreparindent{\parindent=\the\parindent\relax}
\usepackage{parskip}
\restoreparindent

\frenchspacing
\linespread{1.5}
\addto\captionspolish{%
\renewcommand*\listtablename{Spis tabel}
\renewcommand*\tablename{Tabela}
}

\frenchspacing

\begin{document}

\begin{center}

\vspace{1cm}

Studium licencjackie
\end{center}

\vspace{1cm}

\noindent Kierunek: Informatyka

\noindent Specjalność: Bazy danych i technologie internetowe

\vspace{1cm}

{
\leftskip=10cm\noindent
Michał Polakowski\newline
Nr albumu: 203446

}

\vspace{2cm}

\title{Scraping w praktycznych aplikacjach webowych}
\makeatletter

\begin{center}
\LARGE\bf
\@title
\end{center}

\vspace{2cm}

{
\leftskip=10cm\noindent
Praca licencjacka\newline 
napisana w~Instytucie Matematyki, Fizyki i Informatyki\newline
pod kierunkiem naukowym\newline
dra Włodzimierza Bzyla

}

\vfill

\begin{center}
Gdańsk, \the\year
\end{center}
\thispagestyle{empty}

\clearpage
\thispagestyle{empty}
\clearpage

\tableofcontents

\clearpage


\chapter{Wprowadzenie}

Celem niniejszej pracy jest \ldots

\chapter{Django}

Python jest aktualnie jednym z najpopularnijeszych języków programowania na świecie

\section{Popularność}
Pierwszym czynnikiem wpływającym na nasz wybór była popularność frameworka. 

\chapter{Podsumowanie}



\clearpage
\addcontentsline{toc}{chapter}{Bibliografia}
\begin{thebibliography}{99}
\setlength{\itemsep}{0pt}
\end{thebibliography}

\clearpage


\end{document}
